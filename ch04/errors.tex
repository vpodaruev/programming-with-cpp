% !TEX encoding   = UTF8
% !TEX spellcheck = ru_RU
% !TEX root = ../seminars.tex

%%==============
\chapter{Ошибки}
%%==============

%%==============================
\section{Игра <<Быки и Коровы>>}
%%==============================
Разберём вариант реализации игры <<Быки и Коровы>> (см.~упражнение~12 \textbookref{главы~5} учебника).
\cppfile[lastline=4]{projects/04/bulls_and_cows.cpp}

Начнём с~функции \code{main}. Обернём тело в~инструкцию \code{try-catch}, подразумевая обработку ошибок через механизм исключений:

\cppfile[firstline=45, lastline=47]{projects/04/bulls_and_cows.cpp}
\cpp'  // ...'
\cppfile[firstline=80]{projects/04/bulls_and_cows.cpp}

\noindent Фигурные скобки \code{main} опущены намеренно, поскольку тело будет заключено целиком внутрь блока \code{try}.

Размышляя об~игре, сформулируем для~начала правила и выведем на~экран подсказку для~пользователя нашей программки:

\cppfile[firstline=48, lastline=54]{projects/04/bulls_and_cows.cpp}

Укажем число, задуманное компьютером, непосредственно в~программе. Такое решение облегчит первоначальную отладку и сэкономит время разработки первой версии, откладывая размышления о~том, как задать число случайным образом да ещё так, чтобы цифры оказались уникальными. К~этому можно будет вернуться позднее.

Для~удобства доступа к~отдельным цифрам, поскольку их значения как целые числа нам не~нужны, будем хранить число в~виде последовательности символов.

\cppfile[firstline=56, lastline=56]{projects/04/bulls_and_cows.cpp}

Далее, следуя логике игры, пользователь должен попытаться угадать задуманное число. Необходимо дать ему возможность сделать ход и ввести данные. Этот код мы разместим, как отдельное логическое действие, в~функции \code{user\_guess}. А затем подсчитать количество <<быков>> и <<коров>> и вывести эту информацию на~экран:

\cppfile[firstline=58, lastline=77]{projects/04/bulls_and_cows.cpp}

Когда пользователь угадает число, то, очевидно, число <<быков>> станет равным количеству цифр в~задуманном числе, т.\,е. четырём. По~завершению цикла следует напечатать сообщение о~прекращении игры:

\cppfile[firstline=79, lastline=79]{projects/04/bulls_and_cows.cpp}

Теперь рассмотрим код для~вспомогательной функции \code{count}, которая подсчитывает количество цифр в~числе, совпадающих с~указанной.

\cppfile[firstline=6, lastline=15]{projects/04/bulls_and_cows.cpp}

Далее, напишем код, запрашивающий и получающий ввод от~пользователя:

\cppfile[firstline=29, lastline=43]{projects/04/bulls_and_cows.cpp}

Поскольку пользователь может ввести практически всё, что угодно, следует проверить входные данные на~соответствие правилам игры: должно быть четыре различные цифры:

\cppfile[firstline=17, lastline=27]{projects/04/bulls_and_cows.cpp}

Корректные входные данные позволяют нам не~предусматривать дополнительных проверок в~основном цикле функции \code{main}.



%%===================
\subparagraph{P.\,S.}
%%===================
Рассмотренный код "--- лишь основа для~создания хорошей игры. Вспомним хотя бы, что загадывание случайного числа мы отложили на~потом. Попробовав поиграть, становится очевидным, что необходимо вместо завершения программы после обработки исключения как-то возобновить игру. В~конце концов, пользователь мог случайно или по~ошибке нажать клавишу и ввести некорректный ответ. В~число улучшений самого кода следует включить устранение магической константы~4.

Продолжите работу над~игрой, выполняя упражнение~13 из~\textbookref{главы~5}.



%%================
\WhatToReadSection
%%================
\textcite{Stroustrup:2016:ru}: \textbf{глава~6}



%%===============
\ExercisesSection
%%===============
\begin{exercise}
\item Выполните упражнения из \textbookref{главы~5} учебника.
\end{exercise}

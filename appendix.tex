% !TEX encoding   = UTF8
% !TEX spellcheck = ru_RU
% !TEX root = seminars.tex

%%==================
\chapter{Приложение}
%%==================

%%==========================================
\section{Рисование графиков в \lang{Python}}\label{sect:pyplot}
%%==========================================
Каждый язык имеет свои достоинства и недостатки и, как правило, нацелен на эффективное решение определённого класса задач. Совместное использование разных языков часто помогает сократить усилия при разработке программ и повысить гибкость либо удобство создаваемых инструментов. Попробуем продемонстрировать это на примере визуализации результатов обработки данных лабораторной работы по физике, которые можно получить методом наименьших квадратов, рассмотренным в разделе~\ref{chap:ide} (также см. упр.~\ref{ex:plot} на странице~\pageref{ex:plot}).

\begin{figure}[ht]
  {\centering
    \includegraphics[width=0.6\textwidth]{images/line_approx.pdf}

  }
  \caption{Визуализация при помощи \name{Matplotlib}}
  \label{fig:pyplot}
\end{figure}

В языке \lang{C++} нет встроенных графических средств. Для этого необходимо использовать сторонние библиотеки. Работа с подобными инструментами не всегда настолько проста, как хотелось бы. А настройка внешнего вида координатных осей, изображаемых кривых и точек может потребовать перекомпиляции всей программы.

Одним из действительно удобных для этого случая решений является написание сценария (\textenglish{script}) на интерпретируемом языке. \href{\pythonurl}{\lang{Python}} относится к этому ряду языков и имеет мощную поддержку разнообразных средств практически прямо <<из коробки>>. Ниже приведён вариант решения нашей задачи с использованием пакета \href{\matplotliburl}{\name{Matplotlib}}.

\begin{center}
  \yadisk{cpp-seminars/examples/01/plot.py}
\end{center}

\inputminted[linenos, fontsize=\small]{py}{projects/02/plot.py}

Совместить использование разработанных нами отдельных инструментов можно при помощи командной среды. Для этого достаточно всего одной строки (см. страницу~\pageref{sect:shell}):
\begin{consolecode}
$ ./lsm line_approx.txt | xargs python3 plot.py
\end{consolecode}

\noindent Результат в виде графического файла \code{line\_approx.pdf}, изображён на рисунке~\ref{fig:pyplot}.

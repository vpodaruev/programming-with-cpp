% !TEX encoding   = UTF8
% !TEX spellcheck = ru_RU
% !TEX root = ../seminars.tex

%%================
\chapter{Введение}
%%================

%%=====================
\section{Цели и задачи}
%%=====================
\textit{Усвоить основы разработки, тестирования и отладки программ, базовые конструкции языка \lang{C++} и элементы стандартной библиотеки и получить практические навыки их использования при решении ряда простых и более сложных задач.}

%%==========================
\paragraph{Структура курса.}
%%==========================
\begin{itemfeature}
  \item О курсе в целом.
  \item Взаимосвязь с другими курсами.

  \begin{flushleft}\hspace{-4em}\begin{tikzpicture}[node font=\small, >=Stealth]
    \graph [layered layout, components go right top aligned, layer sep=1em]
    {
      AppliedTasks/{Прикладные задачи} [blue, draw=blue, dashed, rounded corners] // [layered layout] {"Аэродинамика", "Динамика полёта", "Прочность", "Проектирование"};

      AppliedSoftware/{Прикладное программное обеспечение} [blue, draw=blue, dashed, rounded corners] // [layered layout, edge=white] {
        OpenFoam, SU2, Gmsh, FreeCAD, ParaView
      };

      OS/{Операционные системы} [blue, draw=blue, dashed, rounded corners, edge=black] // [tree layout] {
        Unix -> {
          Linux -> Android, FreeBSD -> {MacOS, iOS}, Solaris;
        },
        Windows
      };

      ComputerArchitecture/{Архитектура компьютера} [blue, draw=blue, dashed, rounded corners] // [layered layout, edge=<-] {
        "Архитектура и язык ассемблера"
        -> "Микроархитектура"
        -> "<<железо>>"
      };

      ComputerLanguages/{Языки программирования} [blue, draw=blue, dashed, rounded corners] // [layered layout, components go right top aligned, edge={<-, white}] {
        {Python, Java, PHP, HTML, Perl}
        -> {C++, Rust, ObjectiveC[as={Objective C}], Fortran}
        -> {C, Forth, Pascal}
      };

      Algorithms/{Алгоритмы и структуры данных} [blue, draw=blue, dashed, rounded corners] // [layered layout, edge=white] {
        {"Двоичный поиск", "Деревья"}
        -> {"Сортировка", "Хэш-таблицы"}
        -> {"Волновой алгоритм", "Графы"}
      };

      AppliedTasks ->[white] AppliedSoftware ->[white] {OS, ComputerLanguages} ->[white] {ComputerArchitecture, Algorithms};
    };
  \end{tikzpicture}\end{flushleft}

  \item Контрольно-проверочные мероприятия.
\end{itemfeature}



%%===========================
\section{Основная литература}
%%===========================
\cite{Stroustrup:2016:ru}

\nocite{Kernighan:2004:ru, Meyers:2006:ru, Meyers:2000:ru, Meyers:2002:ru, Meyers:2016:ru, Josuttis:2014:ru, Stroustrup:2006:ru, Stroustrup:2013:en}




%%===========================
\section{Материалы и задания}
%%===========================
Значительная часть обучающих упражнений и заданий содержится в книге Бьярне Страуструпа. Дополнительные материалы размещены на \href{\yadiskurl}{яндекс-диске\footnote{\nolinkurl{\yadiskurl}}}:
\begin{itemfeature}
  \item \codebf{books} "--- основная и дополнительная литература;
  \item \codebf{cpp-lectures} "--- лекционные материалы;
  \item \codebf{cpp-seminars} "--- семинарские материалы;
  \begin{itemize}
  	\item \codebf{/examples} "--- исходный код примеров, которые разбираются на занятиях;
  	\item \codebf{/exercises} "--- упражнения с исправлением исходного кода;
  	\item \codebf{/how-to-s} "--- графические инструкции по настройке ПО;
  	\item \codebf{/libraries} "--- библиотеки, которые используются на занятиях;
  	\item \codebf{/program.pdf} "--- программа курса, темы зачёта;
  	\item \codebf{/progress.pdf} "--- планы, успеваемость и проверочные мероприятия;
  	\item \codebf{/seminars.pdf} "--- вспомогательная методичка по материалам занятий.
  \end{itemize}
\end{itemfeature}



%%===============================
\section{Программное обеспечение}
%%===============================
Интегрированная среда разработки:
\begin{itemfeature}
  \item текстовый редактор,
  \item компилятор языка \lang{C++} (с поддержкой стандарта \lang{C++14} или выше),
  \item редактор связей,
  \item средства сборки,
  \item средства отладки.
\end{itemfeature}

Инструкции по установке и настройке рабочей среды размещены в приложении на странице~\pageref{sect:workEnv}.
